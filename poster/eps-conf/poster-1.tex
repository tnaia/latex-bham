\documentclass[a1paper,extrafontsizes,32pt]{a0poster}

\usepackage[utf8]{inputenc}
\usepackage[T1]{fontenc}
\usepackage[english]{babel}
\usepackage{lmodern,,microtype,amssymb}
\usepackage{graphicx}
\usepackage[cymk]{xcolor}
\usepackage{tikz}
\usetikzlibrary{arrows.meta,calc,decorations.markings,decorations.pathmorphing,decorations.pathreplacing,arrows}
\tikzset{lin/.style={line width=5pt,decoration={markings, mark=at position 0.5*\pgfdecoratedpathlength+1.8pt with {\arrow{angle 90}}}, postaction={decorate}}}
\newcommand{\shortendistance}{10pt}
\tikzset{lin shorten/.style={shorten >=\shortendistance,shorten <=\shortendistance,line width=5pt,decoration={markings, mark=at position 0.5*\pgfdecoratedpathlength+1.8pt with {\arrow{angle 90}}}, postaction={decorate}}}
\tikzset{highlight/.style={unav,line width=10pt,line cap=round,line join=round}}
\tikzset{small highlight/.style={orange!80!white,line width=5pt,line cap=round,line join=round}}
\tikzset{small lin/.style={line width=2pt,decoration={markings, mark=at position 0.5*\pgfdecoratedpathlength+1.8pt with {\arrow{angle 90}}}, postaction={decorate}}}
\newcommand{\smallshortendistance}{5pt}
\tikzset{small lin shorten/.style={shorten >=\smallshortendistance,shorten <=\smallshortendistance,line width=2pt,decoration={markings, mark=at position 0.5*\pgfdecoratedpathlength+2.2pt with {\arrow{angle 90}}}, postaction={decorate}}}
\newcommand{\vertexat}[1]{%
  \fill [dark] #1 circle (6pt);}

\usepackage{lmodern}

\renewcommand*\ttdefault{lmvtt}
\renewcommand*\familydefault{\ttdefault} %% Only if the base font of the document is to be typewriter style
\usepackage[T1]{fontenc}

%\usepackage{type1cm}
\usepackage{multicol}
\usepackage[margin=1.5cm]{geometry}
\let\oldemph\emph
\renewcommand{\emph}[1]{\oldemph{\color{red!90!black}#1}}
\newcommand{\diag}[1]{{\sffamily\bfseries\color{blue!80!white}$\langle\,$#1$\,\rangle$}}
\newcommand{\nn}{{\color{dark}\ensuremath{\mathsf{n}}}}
\newcommand{\Fig}{\footnotesize\textcolor{light}{Fig. }}

% Colors
% dark: rgb(28, 42, 63) = cmyk(56%, 33%, 0%, 75%)
%\colorlet{red}[
\definecolor{very dark}{cmyk}{0.56,0.331,0,0.75}
\colorlet{dark}{blue!60!gray}

\colorlet{light}[cmyk]{dark!70!white}
\definecolor{tournament}{cmyk}{0.91,0,0.68,0.35}
%0.851, 0, 0.568, 0.425}
  %0.851, 0, 0.568, 0.51}
  %0.821, 0.11, 0.481, 0.592}
\definecolor{mypink}{cmyk}{0, 1, 0.629, 0}
  %0, 1, 0.752, 0.05}
\colorlet{very light}[cmyk]{dark!30!white}
\colorlet{bgcolor}[cmyk]{white}%{brown!10!white}
\colorlet{red}[cmyk]{red}
\colorlet{unav}[cmyk]{orange!80!red}
\colorlet{green highlight}{green!80!red}
\newcommand{\ang}{} % auxiliar macro
\newcommand{\angtwo}{} % auxiliar macro
\newcommand{\randrotate}{}
\newcommand{\radius}{1.9}
\begin{document}%\sffamily
\pagecolor{bgcolor}
\setlength{\columnsep}{3cm}
\setlength{\parindent}{0em}
\fontsize{32}{36}\selectfont
\setlength{\parskip}{.7\parskip}


{\sffamily\scshape\fontsize{70}{70}\selectfont {\color{unav}Inevitable} patterns}\hfill \raisebox{0ex}{T\'assio Naia\makebox[0cm][l]{ $\rightarrow$}\hspace*{6.2cm}}

  \vspace{-.7cm}{\color{unav}{\rule{56cm}{2pt}}}

  \vspace{.1cm}{\fontsize{35}{40}\selectfont\rmfamily \textit{in}\fontsize{50}{70}\selectfont\ \sffamily\scshape directed networks}\hfill \raisebox{.5ex}{{\color{light}joint work with} Richard Mycroft\hspace*{6.2cm}}
  

     
  


\bigskip

%% \begin{tikzpicture}
%% \foreach \xx in {0,4.5,...,49.5} {
%%   \pgfmathsetmacro{\randrotate}{rand*360}
%%   \begin{scope}[shift={($(\xx cm, 0)$)}]
%%     %  \draw[light] (0,0) circle  (2cm);
%%     \foreach \ii in {0,...,4} {
%%       \pgfmathsetmacro{\ang}{72*\ii +\randrotate}
%%       \fill[dark] (\ang:\radius cm) circle (3pt);
%%       \foreach \jj in {1,2} {
%%         \pgfmathsetmacro{\angtwo}{\ang + 72*\jj)}
%%         \draw [shorten >=6pt, shorten <=6pt,light] (\ang:\radius cm) -- (\angtwo:\radius cm);
%%       }
%%     }
%%   \end{scope}
%% }
%% \end{tikzpicture}
%% \end{center}

\begin{multicols}{2}%
%%
%% \begin{abstract}
%% For reference: Which patterns occur in every round-robin tournament, if there are no ties?
%% A classical result in combinatorics states that (no matter the game results)
%% all players may be lined up so that the first has defeated
%% the second in their match, the second has defeated the third, and so on.
%% In other words, this "directed line" pattern is unavoidable.
%% Interestingly, few such patterns were known, although a conjecture from the 80's 
%% states that most acyclic structures (oriented trees) are unavoidable.
%% Using recently-developed tools in Combinatorics, we are able answer this
%% conjecture, proving that almost all oriented trees are indeed unavoidable.
%% \end{abstract}
%%
\begin{center}
  {\itshape\fontsize{40}{40}\selectfont
    \emph{Directed networks}
  }
  
%%%%%%  \vspace{-.3cm}{\color{unav}{\rule{10.5cm}{2pt}}}
\end{center}

  \medskip

  A natural model for
  
  \hspace*{1em}$\color{light}\hookrightarrow$ electricity, fuel and data distribution,
  
  \hspace*{1em}$\color{light}\hookrightarrow$ algorithm data flow,
  
  \hspace*{1em}$\color{light}\hookrightarrow$ process dependencies in manufacturing,

  \hspace*{1em}$\color{light}\hookrightarrow$ preference systems in biology.
  

  \bigskip

  A \textcolor{tournament}{tournament}
is a network where each pair of \textcolor{dark}{vertices} 
(i.e., \textcolor{dark}{nodes}) is connected by an arc.

\begin{center}  
  \begin{tikzpicture}
    \coordinate (a) at (80:3cm);
    \coordinate (b) at (-40:3cm);
    \coordinate (c) at (200:3cm);
    \coordinate (d) at (0,0)
;

    %    \draw[highlight] (a) --(b) -- (d);
    \begin{scope}[shorten >=.4cm,shorten <=.4cm,tournament]
      \draw[lin,bend left] (a) to node [right] {\makebox[0pt][l]{\textcolor{light}{\reflectbox{$\rightsquigarrow$} arc}}} (b);
      \draw[lin] (b) -- (d);
      \draw[lin] (d) -- (a);
      \draw[lin,bend left] (c) to (a);
      \draw[lin,bend left] (b) to (c);
      \draw[lin] (c) -- (d);
    \end{scope}
    \foreach \ll in {a,b,c} { \fill[dark] (\ll) circle (.3cm); }
    \fill[dark] (d) circle (.3cm);

  \end{tikzpicture}
  %\diag{tournament $(\nn=4)$, define vertices}

  \smallskip
  
  \Fig Tournament on $\nn=4$ vertices.
\end{center}


\medskip

%% There are many tournaments! \newline
%% \hspace*{\fill}(Their number is exponential in \nn.)

%% \medskip

Brute-force analysis of tournaments is impractical,
as the number of tournaments is exponential on \nn.

\bigskip

Despite this variety, it turns out that some
patterns are common to all tournaments. 
The first results in this direction date back to 1934.

\vfill
\columnbreak%
\begin{center}
{\itshape\fontsize{40}{40}\selectfont
\emph{Unavoidability}
}

%%%%%%\vspace{-.4cm}{\color{unav}{\rule{10cm}{2pt}}}
\end{center}

\medskip

A ``structure''  $\bigl(\,$e.g.:
\begin{tikzpicture}
  \coordinate (a) at (0,0);
  \coordinate (b) at (1,.7);
  \coordinate (c) at (2,0);
  \coordinate (d) at (3,.7);

  \draw[small lin shorten,unav] (a) -- (b);
  \draw[small lin shorten,unav] (c) -- (b);
  \draw[small lin shorten,unav] (c) -- (d);

  \foreach \ll in {a,b,c,d} { \fill[dark] (\ll) circle (3pt); }
\end{tikzpicture}$\,\bigr)$
is called \emph{unavoidable}
if it appears in every tournament
(on the same number of vertices).

\bigskip

\emph{Directed cycles} $\Bigl($
%\diag{triangle, square, \ldots})
\begin{tikzpicture}[baseline={(0,-.5)},mypink]
  \coordinate (a) at (170:.8cm);
  \coordinate (b) at (50:.8cm);
  \coordinate (c) at (-70:.8cm);


  \draw[small lin shorten] (a) -- (b);
  \draw[small lin shorten] (b) -- (c);
  \draw[small lin shorten] (c) -- (a);

  \foreach \ll in {a,b,c} { \fill[dark] (\ll) circle (3pt); }
\end{tikzpicture} ,
\begin{tikzpicture}[baseline={(0,-.5)},mypink]
  \coordinate (a) at (160:.9cm);
  \coordinate (b) at (70:.9cm);
  \coordinate (c) at (-20:.9cm);
  \coordinate (d) at (-110:.9cm);


  \draw[small lin shorten] (a) -- (b);
  \draw[small lin shorten] (b) -- (c);
  \draw[small lin shorten] (c) -- (d);
  \draw[small lin shorten] (d) -- (a);

  \foreach \ll in {a,b,c,d} { \fill[dark] (\ll) circle (3pt); }
\end{tikzpicture} , \ldots\ $\Bigr)$
can be \emph{avoided}:

\begin{center}
\begin{tikzpicture}[baseline={(0,-.5)},tournament]
  \coordinate (a) at (170:2.5cm);
  \coordinate (b) at (50:2.5cm);
  \coordinate (c) at (-70:2.5cm);


  \draw[lin shorten] (a) -- (b);
  \draw[lin shorten] (b) -- (c);
  \draw[lin shorten] (a) -- (c);

  \foreach \ll in {a,b,c} { \fill[dark] (\ll) circle (6pt); }
\end{tikzpicture} ,
\begin{tikzpicture}[baseline={(0,-.5)},tournament]
  \coordinate (a) at (0,0);
  \coordinate (b) at (3,0);
  \coordinate (c) at (6,0);
  \coordinate (d) at (9,0);

  \draw[lin shorten] (a) -- (b);
  \draw[lin shorten] (b) -- (c);
  \draw[lin shorten] (c) -- (d);
  \draw[lin shorten,bend right,out=-60,in=-120] (a) to (c);
  \draw[lin shorten,bend left,out=60,in=120] (a) to (d);
  \draw[lin shorten,bend left,out=45,in=135] (b) to (d);
      
  \foreach \ll in {a,b,c,d} { \fill[dark] (\ll) circle (6pt); }
\end{tikzpicture} , $\cdots$

\smallskip
\Fig Avoiding directed cycles.
\end{center}

\bigskip


What about structures without cycles?\newline
\hspace*{\fill} They are called \emph{trees}.
\newcommand{\join}[2]{\draw[lin shorten] (#1)--(#2);}

\begin{center}
  \begin{tikzpicture}
    %% \draw[gray!30!white,very thin] (0,0) grid[step=1cm] (20,6);
    %% \draw[gray!70!white,very thin] (0,0) grid[step=2cm] (20,6);
      \coordinate (a1) at ($(0 +.6*rand,1 +.6*rand)$);
      \coordinate (a2) at ($(1 +.6*rand,3 +.6*rand)$);
      \coordinate (a3) at ($(2 +.6*rand,6 +.6*rand)$);

      \coordinate (b1) at ($(4 +.6*rand,-1 +.6*rand)$);
      \coordinate (b2) at ($(4 +.6*rand,4 +.6*rand)$);
      
      \coordinate (c1) at ($(8 +.6*rand,2 +.6*rand)$);
      \coordinate (c2) at ($(7 +.6*rand,6 +.6*rand)$);

      \coordinate (d1) at ($(10 +.6*rand,0 +.6*rand)$);

      \coordinate (e1) at ($(13 +.6*rand,2 +.6*rand)$);
      \coordinate (e2) at ($(12 +.6*rand,5 +.6*rand)$);

      \coordinate (f1) at ($(15 +.6*rand,0 +.6*rand)$);

      \coordinate (g1) at ($(16 +.6*rand,3 +.6*rand)$);

      \coordinate (h1) at ($(18 +.6*rand,5 +.6*rand)$);
      \foreach \ll in {a1,a2,a3,b1,b2,c1,c2,d1,e1,e2,f1,g1,h1} { \fill[dark] (\ll) circle (6pt); }

      \join{a1}{b1}
      \join{a2}{b2}
      \join{b2}{a3}
      \join{b2}{c2}
      \join{b2}{c1}
      \join{c1}{b1}
      \join{c1}{e2}
      \join{e2}{e1}
      \join{d1}{e1}
      \join{e1}{f1}
      \join{g1}{e2}
      \join{h1}{g1}
    \end{tikzpicture}
\end{center}
\end{multicols}

\bigskip

\begin{center}
%%%%%%  \vspace*{-.4cm}{\color{light}{\rule{49cm}{2pt}}}

  \vspace{-1.4cm}%
  {\color{light}{\rule{\textwidth}{3pt}}}

    \bigskip


  \begin{tikzpicture}
    %%%%%%%%%%%%%%%%%%%%%%%%%%%%%%%%%%%%%%%%%%%%%%%%%%%%%%%%%%%%%%%%%%%%%%%%
%% This is a program to print tournaments version 0.160222
%% Copyright Tássio Naia
%% This is free software GPL 3.0 or later
%%%%%%%%%%%%%%%%%%%%%%%%%%%%%%%%%%%%%%%%%%%%%%%%%%%%%%%%%%%%%%%%%%%%%%%%
% filename: 5v-tourn.txt
% Line 1111111111 has length 10
% I believe the tournaments have 5 vertices.
%%%%%%%%%%%%%%%%%%%%%%%%%%%%%%%%%%%%%%%%%%%%%%%%%%%%%%%%%%%%%%%%%%%%%%%%
% Tournament 1111111111 (number~1)%%%%%%%%%%%%%%%%%%%%%%%%%%%%%%%%%%%%%%
\begin{scope}[xshift=0.00 cm]
\draw [highlight] (277.093:\radius cm) -- (205.093:\radius cm);   % 3 == 2
\draw [highlight] (205.093:\radius cm) -- (133.093:\radius cm);   % 2 == 1
\draw [highlight] (349.093:\radius cm) -- (133.093:\radius cm);   % 4 == 1
\draw [highlight] (133.093:\radius cm) -- ( 61.093:\radius cm);   % 1 == 0
\fill [dark] ( 61.093:\radius cm) circle (3pt);   % vertex 0
\draw [small lin,tournament,shorten >=6pt, shorten <=6pt,tournament] ( 61.093:\radius cm) -- (133.093:\radius cm);   % 0 ->- 1
\draw [small lin,tournament,shorten >=6pt, shorten <=6pt,tournament] ( 61.093:\radius cm) -- (205.093:\radius cm);   % 0 ->- 2
\draw [small lin,tournament,shorten >=6pt, shorten <=6pt,tournament] ( 61.093:\radius cm) -- (277.093:\radius cm);   % 0 ->- 3
\draw [small lin,tournament,shorten >=6pt, shorten <=6pt,tournament] ( 61.093:\radius cm) -- (349.093:\radius cm);   % 0 ->- 4
\fill [dark] (133.093:\radius cm) circle (3pt);   % vertex 1
\draw [small lin,tournament,shorten >=6pt, shorten <=6pt,tournament] (133.093:\radius cm) -- (205.093:\radius cm);   % 1 ->- 2
\draw [small lin,tournament,shorten >=6pt, shorten <=6pt,tournament] (133.093:\radius cm) -- (277.093:\radius cm);   % 1 ->- 3
\draw [small lin,tournament,shorten >=6pt, shorten <=6pt,tournament] (133.093:\radius cm) -- (349.093:\radius cm);   % 1 ->- 4
\fill [dark] (205.093:\radius cm) circle (3pt);   % vertex 2
\draw [small lin,tournament,shorten >=6pt, shorten <=6pt,tournament] (205.093:\radius cm) -- (277.093:\radius cm);   % 2 ->- 3
\draw [small lin,tournament,shorten >=6pt, shorten <=6pt,tournament] (205.093:\radius cm) -- (349.093:\radius cm);   % 2 ->- 4
\fill [dark] (277.093:\radius cm) circle (3pt);   % vertex 3
\draw [small lin,tournament,shorten >=6pt, shorten <=6pt,tournament] (277.093:\radius cm) -- (349.093:\radius cm);   % 3 ->- 4
\fill [dark] (349.093:\radius cm) circle (3pt);   % vertex 4
\end{scope}%
% Tournament 1111111101 (number~2)%%%%%%%%%%%%%%%%%%%%%%%%%%%%%%%%%%%%%%
\begin{scope}[xshift=4.50 cm]
\draw [highlight] (150.325:\radius cm) -- (294.325:\radius cm);   % 2 == 4
\draw [highlight] (294.325:\radius cm) -- ( 78.325:\radius cm);   % 4 == 1
\draw [highlight] (222.325:\radius cm) -- ( 78.325:\radius cm);   % 3 == 1
\draw [highlight] ( 78.325:\radius cm) -- (  6.325:\radius cm);   % 1 == 0
\fill [dark] (  6.325:\radius cm) circle (3pt);   % vertex 0
\draw [small lin,tournament,shorten >=6pt, shorten <=6pt,tournament] (  6.325:\radius cm) -- ( 78.325:\radius cm);   % 0 ->- 1
\draw [small lin,tournament,shorten >=6pt, shorten <=6pt,tournament] (  6.325:\radius cm) -- (150.325:\radius cm);   % 0 ->- 2
\draw [small lin,tournament,shorten >=6pt, shorten <=6pt,tournament] (  6.325:\radius cm) -- (222.325:\radius cm);   % 0 ->- 3
\draw [small lin,tournament,shorten >=6pt, shorten <=6pt,tournament] (  6.325:\radius cm) -- (294.325:\radius cm);   % 0 ->- 4
\fill [dark] ( 78.325:\radius cm) circle (3pt);   % vertex 1
\draw [small lin,tournament,shorten >=6pt, shorten <=6pt,tournament] ( 78.325:\radius cm) -- (150.325:\radius cm);   % 1 ->- 2
\draw [small lin,tournament,shorten >=6pt, shorten <=6pt,tournament] ( 78.325:\radius cm) -- (222.325:\radius cm);   % 1 ->- 3
\draw [small lin,tournament,shorten >=6pt, shorten <=6pt,tournament] ( 78.325:\radius cm) -- (294.325:\radius cm);   % 1 ->- 4
\fill [dark] (150.325:\radius cm) circle (3pt);   % vertex 2
\draw [small lin,tournament,shorten >=6pt, shorten <=6pt,tournament] (150.325:\radius cm) -- (222.325:\radius cm);   % 2 ->- 3
\draw [small lin,tournament,shorten >=6pt, shorten <=6pt,tournament] (294.325:\radius cm) -- (150.325:\radius cm);   % 4 ->- 2
\fill [dark] (222.325:\radius cm) circle (3pt);   % vertex 3
\draw [small lin,tournament,shorten >=6pt, shorten <=6pt,tournament] (222.325:\radius cm) -- (294.325:\radius cm);   % 3 ->- 4
\fill [dark] (294.325:\radius cm) circle (3pt);   % vertex 4
\end{scope}%
% Tournament 1101111111 (number~3)%%%%%%%%%%%%%%%%%%%%%%%%%%%%%%%%%%%%%%
\begin{scope}[xshift=9.00 cm]
\draw [highlight] ( 21.573:\radius cm) -- (237.573:\radius cm);   % 0 == 3
\draw [highlight] (237.573:\radius cm) -- (165.573:\radius cm);   % 3 == 2
\draw [highlight] (309.573:\radius cm) -- (165.573:\radius cm);   % 4 == 2
\draw [highlight] (165.573:\radius cm) -- ( 93.573:\radius cm);   % 2 == 1
\fill [dark] ( 21.573:\radius cm) circle (3pt);   % vertex 0
\draw [small lin,tournament,shorten >=6pt, shorten <=6pt,tournament] ( 21.573:\radius cm) -- ( 93.573:\radius cm);   % 0 ->- 1
\draw [small lin,tournament,shorten >=6pt, shorten <=6pt,tournament] ( 21.573:\radius cm) -- (165.573:\radius cm);   % 0 ->- 2
\draw [small lin,tournament,shorten >=6pt, shorten <=6pt,tournament] (237.573:\radius cm) -- ( 21.573:\radius cm);   % 3 ->- 0
\draw [small lin,tournament,shorten >=6pt, shorten <=6pt,tournament] ( 21.573:\radius cm) -- (309.573:\radius cm);   % 0 ->- 4
\fill [dark] ( 93.573:\radius cm) circle (3pt);   % vertex 1
\draw [small lin,tournament,shorten >=6pt, shorten <=6pt,tournament] ( 93.573:\radius cm) -- (165.573:\radius cm);   % 1 ->- 2
\draw [small lin,tournament,shorten >=6pt, shorten <=6pt,tournament] ( 93.573:\radius cm) -- (237.573:\radius cm);   % 1 ->- 3
\draw [small lin,tournament,shorten >=6pt, shorten <=6pt,tournament] ( 93.573:\radius cm) -- (309.573:\radius cm);   % 1 ->- 4
\fill [dark] (165.573:\radius cm) circle (3pt);   % vertex 2
\draw [small lin,tournament,shorten >=6pt, shorten <=6pt,tournament] (165.573:\radius cm) -- (237.573:\radius cm);   % 2 ->- 3
\draw [small lin,tournament,shorten >=6pt, shorten <=6pt,tournament] (165.573:\radius cm) -- (309.573:\radius cm);   % 2 ->- 4
\fill [dark] (237.573:\radius cm) circle (3pt);   % vertex 3
\draw [small lin,tournament,shorten >=6pt, shorten <=6pt,tournament] (237.573:\radius cm) -- (309.573:\radius cm);   % 3 ->- 4
\fill [dark] (309.573:\radius cm) circle (3pt);   % vertex 4
\end{scope}%
% Tournament 1100111111 (number~4)%%%%%%%%%%%%%%%%%%%%%%%%%%%%%%%%%%%%%%
\begin{scope}[xshift=13.50 cm]
\draw [highlight] ( 64.547:\radius cm) -- (280.547:\radius cm);   % 0 == 3
\draw [highlight] (280.547:\radius cm) -- (208.547:\radius cm);   % 3 == 2
\draw [highlight] (352.547:\radius cm) -- (208.547:\radius cm);   % 4 == 2
\draw [highlight] (208.547:\radius cm) -- (136.547:\radius cm);   % 2 == 1
\fill [dark] ( 64.547:\radius cm) circle (3pt);   % vertex 0
\draw [small lin,tournament,shorten >=6pt, shorten <=6pt,tournament] ( 64.547:\radius cm) -- (136.547:\radius cm);   % 0 ->- 1
\draw [small lin,tournament,shorten >=6pt, shorten <=6pt,tournament] ( 64.547:\radius cm) -- (208.547:\radius cm);   % 0 ->- 2
\draw [small lin,tournament,shorten >=6pt, shorten <=6pt,tournament] (280.547:\radius cm) -- ( 64.547:\radius cm);   % 3 ->- 0
\draw [small lin,tournament,shorten >=6pt, shorten <=6pt,tournament] (352.547:\radius cm) -- ( 64.547:\radius cm);   % 4 ->- 0
\fill [dark] (136.547:\radius cm) circle (3pt);   % vertex 1
\draw [small lin,tournament,shorten >=6pt, shorten <=6pt,tournament] (136.547:\radius cm) -- (208.547:\radius cm);   % 1 ->- 2
\draw [small lin,tournament,shorten >=6pt, shorten <=6pt,tournament] (136.547:\radius cm) -- (280.547:\radius cm);   % 1 ->- 3
\draw [small lin,tournament,shorten >=6pt, shorten <=6pt,tournament] (136.547:\radius cm) -- (352.547:\radius cm);   % 1 ->- 4
\fill [dark] (208.547:\radius cm) circle (3pt);   % vertex 2
\draw [small lin,tournament,shorten >=6pt, shorten <=6pt,tournament] (208.547:\radius cm) -- (280.547:\radius cm);   % 2 ->- 3
\draw [small lin,tournament,shorten >=6pt, shorten <=6pt,tournament] (208.547:\radius cm) -- (352.547:\radius cm);   % 2 ->- 4
\fill [dark] (280.547:\radius cm) circle (3pt);   % vertex 3
\draw [small lin,tournament,shorten >=6pt, shorten <=6pt,tournament] (280.547:\radius cm) -- (352.547:\radius cm);   % 3 ->- 4
\fill [dark] (352.547:\radius cm) circle (3pt);   % vertex 4
\end{scope}%
% Tournament 1101110111 (number~5)%%%%%%%%%%%%%%%%%%%%%%%%%%%%%%%%%%%%%%
\begin{scope}[xshift=18.00 cm]
\draw [highlight] ( 47.813:\radius cm) -- (263.813:\radius cm);   % 0 == 3
\draw [highlight] (263.813:\radius cm) -- (119.813:\radius cm);   % 3 == 1
\draw [highlight] (191.813:\radius cm) -- (119.813:\radius cm);   % 2 == 1
\draw [highlight] (119.813:\radius cm) -- (335.813:\radius cm);   % 1 == 4
\fill [dark] ( 47.813:\radius cm) circle (3pt);   % vertex 0
\draw [small lin,tournament,shorten >=6pt, shorten <=6pt,tournament] ( 47.813:\radius cm) -- (119.813:\radius cm);   % 0 ->- 1
\draw [small lin,tournament,shorten >=6pt, shorten <=6pt,tournament] ( 47.813:\radius cm) -- (191.813:\radius cm);   % 0 ->- 2
\draw [small lin,tournament,shorten >=6pt, shorten <=6pt,tournament] (263.813:\radius cm) -- ( 47.813:\radius cm);   % 3 ->- 0
\draw [small lin,tournament,shorten >=6pt, shorten <=6pt,tournament] ( 47.813:\radius cm) -- (335.813:\radius cm);   % 0 ->- 4
\fill [dark] (119.813:\radius cm) circle (3pt);   % vertex 1
\draw [small lin,tournament,shorten >=6pt, shorten <=6pt,tournament] (119.813:\radius cm) -- (191.813:\radius cm);   % 1 ->- 2
\draw [small lin,tournament,shorten >=6pt, shorten <=6pt,tournament] (119.813:\radius cm) -- (263.813:\radius cm);   % 1 ->- 3
\draw [small lin,tournament,shorten >=6pt, shorten <=6pt,tournament] (335.813:\radius cm) -- (119.813:\radius cm);   % 4 ->- 1
\fill [dark] (191.813:\radius cm) circle (3pt);   % vertex 2
\draw [small lin,tournament,shorten >=6pt, shorten <=6pt,tournament] (191.813:\radius cm) -- (263.813:\radius cm);   % 2 ->- 3
\draw [small lin,tournament,shorten >=6pt, shorten <=6pt,tournament] (191.813:\radius cm) -- (335.813:\radius cm);   % 2 ->- 4
\fill [dark] (263.813:\radius cm) circle (3pt);   % vertex 3
\draw [small lin,tournament,shorten >=6pt, shorten <=6pt,tournament] (263.813:\radius cm) -- (335.813:\radius cm);   % 3 ->- 4
\fill [dark] (335.813:\radius cm) circle (3pt);   % vertex 4
\end{scope}%
% Tournament 1101111101 (number~6)%%%%%%%%%%%%%%%%%%%%%%%%%%%%%%%%%%%%%%
\begin{scope}[xshift=22.50 cm]
\draw [highlight] ( 95.841:\radius cm) -- ( 23.841:\radius cm);   % 1 == 0
\draw [highlight] ( 23.841:\radius cm) -- (239.841:\radius cm);   % 0 == 3
\draw [highlight] (311.841:\radius cm) -- (239.841:\radius cm);   % 4 == 3
\draw [highlight] (239.841:\radius cm) -- (167.841:\radius cm);   % 3 == 2
\fill [dark] ( 23.841:\radius cm) circle (3pt);   % vertex 0
\draw [small lin,tournament,shorten >=6pt, shorten <=6pt,tournament] ( 23.841:\radius cm) -- ( 95.841:\radius cm);   % 0 ->- 1
\draw [small lin,tournament,shorten >=6pt, shorten <=6pt,tournament] ( 23.841:\radius cm) -- (167.841:\radius cm);   % 0 ->- 2
\draw [small lin,tournament,shorten >=6pt, shorten <=6pt,tournament] (239.841:\radius cm) -- ( 23.841:\radius cm);   % 3 ->- 0
\draw [small lin,tournament,shorten >=6pt, shorten <=6pt,tournament] ( 23.841:\radius cm) -- (311.841:\radius cm);   % 0 ->- 4
\fill [dark] ( 95.841:\radius cm) circle (3pt);   % vertex 1
\draw [small lin,tournament,shorten >=6pt, shorten <=6pt,tournament] ( 95.841:\radius cm) -- (167.841:\radius cm);   % 1 ->- 2
\draw [small lin,tournament,shorten >=6pt, shorten <=6pt,tournament] ( 95.841:\radius cm) -- (239.841:\radius cm);   % 1 ->- 3
\draw [small lin,tournament,shorten >=6pt, shorten <=6pt,tournament] ( 95.841:\radius cm) -- (311.841:\radius cm);   % 1 ->- 4
\fill [dark] (167.841:\radius cm) circle (3pt);   % vertex 2
\draw [small lin,tournament,shorten >=6pt, shorten <=6pt,tournament] (167.841:\radius cm) -- (239.841:\radius cm);   % 2 ->- 3
\draw [small lin,tournament,shorten >=6pt, shorten <=6pt,tournament] (311.841:\radius cm) -- (167.841:\radius cm);   % 4 ->- 2
\fill [dark] (239.841:\radius cm) circle (3pt);   % vertex 3
\draw [small lin,tournament,shorten >=6pt, shorten <=6pt,tournament] (239.841:\radius cm) -- (311.841:\radius cm);   % 3 ->- 4
\fill [dark] (311.841:\radius cm) circle (3pt);   % vertex 4
\end{scope}%
% Tournament 1101111110 (number~7)%%%%%%%%%%%%%%%%%%%%%%%%%%%%%%%%%%%%%%
\begin{scope}[xshift=27.00 cm]
\draw [highlight] ( 30.462:\radius cm) -- (246.462:\radius cm);   % 0 == 3
\draw [highlight] (246.462:\radius cm) -- (174.462:\radius cm);   % 3 == 2
\draw [highlight] (318.462:\radius cm) -- (174.462:\radius cm);   % 4 == 2
\draw [highlight] (174.462:\radius cm) -- (102.462:\radius cm);   % 2 == 1
\fill [dark] ( 30.462:\radius cm) circle (3pt);   % vertex 0
\draw [small lin,tournament,shorten >=6pt, shorten <=6pt,tournament] ( 30.462:\radius cm) -- (102.462:\radius cm);   % 0 ->- 1
\draw [small lin,tournament,shorten >=6pt, shorten <=6pt,tournament] ( 30.462:\radius cm) -- (174.462:\radius cm);   % 0 ->- 2
\draw [small lin,tournament,shorten >=6pt, shorten <=6pt,tournament] (246.462:\radius cm) -- ( 30.462:\radius cm);   % 3 ->- 0
\draw [small lin,tournament,shorten >=6pt, shorten <=6pt,tournament] ( 30.462:\radius cm) -- (318.462:\radius cm);   % 0 ->- 4
\fill [dark] (102.462:\radius cm) circle (3pt);   % vertex 1
\draw [small lin,tournament,shorten >=6pt, shorten <=6pt,tournament] (102.462:\radius cm) -- (174.462:\radius cm);   % 1 ->- 2
\draw [small lin,tournament,shorten >=6pt, shorten <=6pt,tournament] (102.462:\radius cm) -- (246.462:\radius cm);   % 1 ->- 3
\draw [small lin,tournament,shorten >=6pt, shorten <=6pt,tournament] (102.462:\radius cm) -- (318.462:\radius cm);   % 1 ->- 4
\fill [dark] (174.462:\radius cm) circle (3pt);   % vertex 2
\draw [small lin,tournament,shorten >=6pt, shorten <=6pt,tournament] (174.462:\radius cm) -- (246.462:\radius cm);   % 2 ->- 3
\draw [small lin,tournament,shorten >=6pt, shorten <=6pt,tournament] (174.462:\radius cm) -- (318.462:\radius cm);   % 2 ->- 4
\fill [dark] (246.462:\radius cm) circle (3pt);   % vertex 3
\draw [small lin,tournament,shorten >=6pt, shorten <=6pt,tournament] (318.462:\radius cm) -- (246.462:\radius cm);   % 4 ->- 3
\fill [dark] (318.462:\radius cm) circle (3pt);   % vertex 4
\end{scope}%
% Tournament 1100110111 (number~8)%%%%%%%%%%%%%%%%%%%%%%%%%%%%%%%%%%%%%%
\begin{scope}[xshift=31.50 cm]
\draw [highlight] ( 12.162:\radius cm) -- (228.162:\radius cm);   % 0 == 3
\draw [highlight] (228.162:\radius cm) -- ( 84.162:\radius cm);   % 3 == 1
\draw [highlight] (156.162:\radius cm) -- ( 84.162:\radius cm);   % 2 == 1
\draw [highlight] ( 84.162:\radius cm) -- (300.162:\radius cm);   % 1 == 4
\fill [dark] ( 12.162:\radius cm) circle (3pt);   % vertex 0
\draw [small lin,tournament,shorten >=6pt, shorten <=6pt,tournament] ( 12.162:\radius cm) -- ( 84.162:\radius cm);   % 0 ->- 1
\draw [small lin,tournament,shorten >=6pt, shorten <=6pt,tournament] ( 12.162:\radius cm) -- (156.162:\radius cm);   % 0 ->- 2
\draw [small lin,tournament,shorten >=6pt, shorten <=6pt,tournament] (228.162:\radius cm) -- ( 12.162:\radius cm);   % 3 ->- 0
\draw [small lin,tournament,shorten >=6pt, shorten <=6pt,tournament] (300.162:\radius cm) -- ( 12.162:\radius cm);   % 4 ->- 0
\fill [dark] ( 84.162:\radius cm) circle (3pt);   % vertex 1
\draw [small lin,tournament,shorten >=6pt, shorten <=6pt,tournament] ( 84.162:\radius cm) -- (156.162:\radius cm);   % 1 ->- 2
\draw [small lin,tournament,shorten >=6pt, shorten <=6pt,tournament] ( 84.162:\radius cm) -- (228.162:\radius cm);   % 1 ->- 3
\draw [small lin,tournament,shorten >=6pt, shorten <=6pt,tournament] (300.162:\radius cm) -- ( 84.162:\radius cm);   % 4 ->- 1
\fill [dark] (156.162:\radius cm) circle (3pt);   % vertex 2
\draw [small lin,tournament,shorten >=6pt, shorten <=6pt,tournament] (156.162:\radius cm) -- (228.162:\radius cm);   % 2 ->- 3
\draw [small lin,tournament,shorten >=6pt, shorten <=6pt,tournament] (156.162:\radius cm) -- (300.162:\radius cm);   % 2 ->- 4
\fill [dark] (228.162:\radius cm) circle (3pt);   % vertex 3
\draw [small lin,tournament,shorten >=6pt, shorten <=6pt,tournament] (228.162:\radius cm) -- (300.162:\radius cm);   % 3 ->- 4
\fill [dark] (300.162:\radius cm) circle (3pt);   % vertex 4
\end{scope}%
% Tournament 1111101111 (number~9)%%%%%%%%%%%%%%%%%%%%%%%%%%%%%%%%%%%%%%
\begin{scope}[xshift=36.00 cm]
\draw [highlight] (107.114:\radius cm) -- (251.114:\radius cm);   % 1 == 3
\draw [highlight] (251.114:\radius cm) -- (179.114:\radius cm);   % 3 == 2
\draw [highlight] (323.114:\radius cm) -- (179.114:\radius cm);   % 4 == 2
\draw [highlight] (179.114:\radius cm) -- ( 35.114:\radius cm);   % 2 == 0
\fill [dark] ( 35.114:\radius cm) circle (3pt);   % vertex 0
\draw [small lin,tournament,shorten >=6pt, shorten <=6pt,tournament] ( 35.114:\radius cm) -- (107.114:\radius cm);   % 0 ->- 1
\draw [small lin,tournament,shorten >=6pt, shorten <=6pt,tournament] ( 35.114:\radius cm) -- (179.114:\radius cm);   % 0 ->- 2
\draw [small lin,tournament,shorten >=6pt, shorten <=6pt,tournament] ( 35.114:\radius cm) -- (251.114:\radius cm);   % 0 ->- 3
\draw [small lin,tournament,shorten >=6pt, shorten <=6pt,tournament] ( 35.114:\radius cm) -- (323.114:\radius cm);   % 0 ->- 4
\fill [dark] (107.114:\radius cm) circle (3pt);   % vertex 1
\draw [small lin,tournament,shorten >=6pt, shorten <=6pt,tournament] (107.114:\radius cm) -- (179.114:\radius cm);   % 1 ->- 2
\draw [small lin,tournament,shorten >=6pt, shorten <=6pt,tournament] (251.114:\radius cm) -- (107.114:\radius cm);   % 3 ->- 1
\draw [small lin,tournament,shorten >=6pt, shorten <=6pt,tournament] (107.114:\radius cm) -- (323.114:\radius cm);   % 1 ->- 4
\fill [dark] (179.114:\radius cm) circle (3pt);   % vertex 2
\draw [small lin,tournament,shorten >=6pt, shorten <=6pt,tournament] (179.114:\radius cm) -- (251.114:\radius cm);   % 2 ->- 3
\draw [small lin,tournament,shorten >=6pt, shorten <=6pt,tournament] (179.114:\radius cm) -- (323.114:\radius cm);   % 2 ->- 4
\fill [dark] (251.114:\radius cm) circle (3pt);   % vertex 3
\draw [small lin,tournament,shorten >=6pt, shorten <=6pt,tournament] (251.114:\radius cm) -- (323.114:\radius cm);   % 3 ->- 4
\fill [dark] (323.114:\radius cm) circle (3pt);   % vertex 4
\end{scope}%
% Tournament 1110101111 (number~10)%%%%%%%%%%%%%%%%%%%%%%%%%%%%%%%%%%%%%
\begin{scope}[xshift=40.50 cm]
\draw [highlight] ( 21.614:\radius cm) -- (309.614:\radius cm);   % 0 == 4
\draw [highlight] (309.614:\radius cm) -- ( 93.614:\radius cm);   % 4 == 1
\draw [highlight] (165.614:\radius cm) -- ( 93.614:\radius cm);   % 2 == 1
\draw [highlight] ( 93.614:\radius cm) -- (237.614:\radius cm);   % 1 == 3
\fill [dark] ( 21.614:\radius cm) circle (3pt);   % vertex 0
\draw [small lin,tournament,shorten >=6pt, shorten <=6pt,tournament] ( 21.614:\radius cm) -- ( 93.614:\radius cm);   % 0 ->- 1
\draw [small lin,tournament,shorten >=6pt, shorten <=6pt,tournament] ( 21.614:\radius cm) -- (165.614:\radius cm);   % 0 ->- 2
\draw [small lin,tournament,shorten >=6pt, shorten <=6pt,tournament] ( 21.614:\radius cm) -- (237.614:\radius cm);   % 0 ->- 3
\draw [small lin,tournament,shorten >=6pt, shorten <=6pt,tournament] (309.614:\radius cm) -- ( 21.614:\radius cm);   % 4 ->- 0
\fill [dark] ( 93.614:\radius cm) circle (3pt);   % vertex 1
\draw [small lin,tournament,shorten >=6pt, shorten <=6pt,tournament] ( 93.614:\radius cm) -- (165.614:\radius cm);   % 1 ->- 2
\draw [small lin,tournament,shorten >=6pt, shorten <=6pt,tournament] (237.614:\radius cm) -- ( 93.614:\radius cm);   % 3 ->- 1
\draw [small lin,tournament,shorten >=6pt, shorten <=6pt,tournament] ( 93.614:\radius cm) -- (309.614:\radius cm);   % 1 ->- 4
\fill [dark] (165.614:\radius cm) circle (3pt);   % vertex 2
\draw [small lin,tournament,shorten >=6pt, shorten <=6pt,tournament] (165.614:\radius cm) -- (237.614:\radius cm);   % 2 ->- 3
\draw [small lin,tournament,shorten >=6pt, shorten <=6pt,tournament] (165.614:\radius cm) -- (309.614:\radius cm);   % 2 ->- 4
\fill [dark] (237.614:\radius cm) circle (3pt);   % vertex 3
\draw [small lin,tournament,shorten >=6pt, shorten <=6pt,tournament] (237.614:\radius cm) -- (309.614:\radius cm);   % 3 ->- 4
\fill [dark] (309.614:\radius cm) circle (3pt);   % vertex 4
\end{scope}%
% Tournament 1111100111 (number~11)%%%%%%%%%%%%%%%%%%%%%%%%%%%%%%%%%%%%%
\begin{scope}[xshift=45.00 cm]
\draw [highlight] ( 78.734:\radius cm) -- (222.734:\radius cm);   % 1 == 3
\draw [highlight] (222.734:\radius cm) -- (150.734:\radius cm);   % 3 == 2
\draw [highlight] (294.734:\radius cm) -- (150.734:\radius cm);   % 4 == 2
\draw [highlight] (150.734:\radius cm) -- (  6.734:\radius cm);   % 2 == 0
\fill [dark] (  6.734:\radius cm) circle (3pt);   % vertex 0
\draw [small lin,tournament,shorten >=6pt, shorten <=6pt,tournament] (  6.734:\radius cm) -- ( 78.734:\radius cm);   % 0 ->- 1
\draw [small lin,tournament,shorten >=6pt, shorten <=6pt,tournament] (  6.734:\radius cm) -- (150.734:\radius cm);   % 0 ->- 2
\draw [small lin,tournament,shorten >=6pt, shorten <=6pt,tournament] (  6.734:\radius cm) -- (222.734:\radius cm);   % 0 ->- 3
\draw [small lin,tournament,shorten >=6pt, shorten <=6pt,tournament] (  6.734:\radius cm) -- (294.734:\radius cm);   % 0 ->- 4
\fill [dark] ( 78.734:\radius cm) circle (3pt);   % vertex 1
\draw [small lin,tournament,shorten >=6pt, shorten <=6pt,tournament] ( 78.734:\radius cm) -- (150.734:\radius cm);   % 1 ->- 2
\draw [small lin,tournament,shorten >=6pt, shorten <=6pt,tournament] (222.734:\radius cm) -- ( 78.734:\radius cm);   % 3 ->- 1
\draw [small lin,tournament,shorten >=6pt, shorten <=6pt,tournament] (294.734:\radius cm) -- ( 78.734:\radius cm);   % 4 ->- 1
\fill [dark] (150.734:\radius cm) circle (3pt);   % vertex 2
\draw [small lin,tournament,shorten >=6pt, shorten <=6pt,tournament] (150.734:\radius cm) -- (222.734:\radius cm);   % 2 ->- 3
\draw [small lin,tournament,shorten >=6pt, shorten <=6pt,tournament] (150.734:\radius cm) -- (294.734:\radius cm);   % 2 ->- 4
\fill [dark] (222.734:\radius cm) circle (3pt);   % vertex 3
\draw [small lin,tournament,shorten >=6pt, shorten <=6pt,tournament] (222.734:\radius cm) -- (294.734:\radius cm);   % 3 ->- 4
\fill [dark] (294.734:\radius cm) circle (3pt);   % vertex 4
\end{scope}%
% Tournament 1011111111 (number~12)%%%%%%%%%%%%%%%%%%%%%%%%%%%%%%%%%%%%%
\begin{scope}[xshift=49.50 cm]
\draw [highlight] (256.401:\radius cm) -- ( 40.401:\radius cm);   % 3 == 0
\draw [highlight] ( 40.401:\radius cm) -- (184.401:\radius cm);   % 0 == 2
\draw [highlight] (328.401:\radius cm) -- (184.401:\radius cm);   % 4 == 2
\draw [highlight] (184.401:\radius cm) -- (112.401:\radius cm);   % 2 == 1
\fill [dark] ( 40.401:\radius cm) circle (3pt);   % vertex 0
\draw [small lin,tournament,shorten >=6pt, shorten <=6pt,tournament] ( 40.401:\radius cm) -- (112.401:\radius cm);   % 0 ->- 1
\draw [small lin,tournament,shorten >=6pt, shorten <=6pt,tournament] (184.401:\radius cm) -- ( 40.401:\radius cm);   % 2 ->- 0
\draw [small lin,tournament,shorten >=6pt, shorten <=6pt,tournament] ( 40.401:\radius cm) -- (256.401:\radius cm);   % 0 ->- 3
\draw [small lin,tournament,shorten >=6pt, shorten <=6pt,tournament] ( 40.401:\radius cm) -- (328.401:\radius cm);   % 0 ->- 4
\fill [dark] (112.401:\radius cm) circle (3pt);   % vertex 1
\draw [small lin,tournament,shorten >=6pt, shorten <=6pt,tournament] (112.401:\radius cm) -- (184.401:\radius cm);   % 1 ->- 2
\draw [small lin,tournament,shorten >=6pt, shorten <=6pt,tournament] (112.401:\radius cm) -- (256.401:\radius cm);   % 1 ->- 3
\draw [small lin,tournament,shorten >=6pt, shorten <=6pt,tournament] (112.401:\radius cm) -- (328.401:\radius cm);   % 1 ->- 4
\fill [dark] (184.401:\radius cm) circle (3pt);   % vertex 2
\draw [small lin,tournament,shorten >=6pt, shorten <=6pt,tournament] (184.401:\radius cm) -- (256.401:\radius cm);   % 2 ->- 3
\draw [small lin,tournament,shorten >=6pt, shorten <=6pt,tournament] (184.401:\radius cm) -- (328.401:\radius cm);   % 2 ->- 4
\fill [dark] (256.401:\radius cm) circle (3pt);   % vertex 3
\draw [small lin,tournament,shorten >=6pt, shorten <=6pt,tournament] (256.401:\radius cm) -- (328.401:\radius cm);   % 3 ->- 4
\fill [dark] (328.401:\radius cm) circle (3pt);   % vertex 4
\end{scope}%

    %% \foreach \xx in {0,4.5,...,49.5} {
    %%   \pgfmathsetmacro{\randrotate}{rand*360}
    %%   \begin{scope}[shift={($(\xx cm, 0)$)}]
    %%     %  \draw[light] (0,0) circle  (2cm);
    %%     \foreach \ii in {0,...,4} {
    %%       \pgfmathsetmacro{\ang}{72*\ii +\randrotate}
    %%       \fill[dark] (\ang:\radius cm) circle (3pt);
    %%       \foreach \jj in {1,2} {
    %%         \pgfmathsetmacro{\angtwo}{\ang + 72*\jj)}
    %%         \draw [shorten >=6pt, shorten <=6pt,light] (\ang:\radius cm) -- (\angtwo:\radius cm);
    %%       }
    %%     }
    %%   \end{scope}
    %% }
  \end{tikzpicture}

  \smallskip

  %% % Raised Rule Command:
  %% % Arg 1 (Optional) - how high to raise the rule
  %% % Arg 2            - Thickness of the rule
  %% \newcommand{\raisedrule}[2][0em]{\leaders\hbox{\rule[#1]{1pt}{#2}}\hfill\null}

  %% \def\Vrulefill#1#2{
  %%   \leavevmode%
  %%   \hskip-.2in%
  %%   \leaders%
  %%   \vtop{\hsize=.0025in\vskip#1#2}%
  %%   \hfill%
  %%   \hskip.3in%
  %% }

  {\color{light}\rule[.3ex]{32cm}{3pt}} \hfill\Fig The tree pattern
  \begin{tikzpicture}[baseline={(0,0)}]
    \coordinate (a1) at ($(2.5em,0) +(200:2.5em)$);
    \coordinate (a2) at ($(2.5em,0) +(160:2.5em)$);
    \coordinate (b)  at (2.5em,0em);
    \coordinate (c)  at ($(b) +(20:2.5em)$);
    \coordinate (d)  at ($(c) +(-40:2.5em)$);%(5em,1.3em);
    \draw[small lin shorten,unav] (a1) -- (b);
    \draw[small lin shorten,unav] (a2) -- (b);
    \draw[small lin shorten,unav] (b) -- (c);
    \draw[small lin shorten,unav] (c) -- (d);
    \foreach \ll in {a1,a2,b,c,d} { \fill[dark] (\ll) circle (3pt);}
    
  \end{tikzpicture}\hfill
  appears in every \textcolor{dark}{5}-vertex tournament. {\color{light}\rule[.5ex]{2em}{3pt}}
  \bigskip

  %\vspace{-1.4cm}%
  %{\color{light}{\rule{\textwidth}{3pt}}}

%%%%%%  %\vspace{-1.4cm}{\color{unav}{\rule{49cm}{2pt}}}

\end{center}

\medskip

\begin{multicols}{2}
\vfill
\columnbreak%
\begin{center}
{\itshape\fontsize{40}{40}\selectfont
\emph{Which trees are unavoidable\makebox[0em][l]{?}}
}

%%%%%%\vspace{-.4cm}{\color{unav}{\rule{17cm}{2pt}}}
\end{center}

\bigskip

Only two known examples: \emph{paths} ($\nn\geq 8$) and \emph{claws}.

\bigskip

\begin{center}

  \begin{tikzpicture}[baseline={(0,-6)}]
    \coordinate (a) at (0,0);
    \coordinate (b) at ($(a) + (0,-3)$);
    \coordinate (c) at ($(b) + (0,-3)$);
    \coordinate (d) at ($(c) + (0,-3)$);
    \coordinate (e) at ($(d) + (0,-3)$);

    \begin{scope}[unav]
    \draw [lin shorten] (b) -- (a);
    \draw [lin shorten] (b) -- (c);
    \draw [lin shorten] (c) -- (d);
    %\draw [lin shorten] (e) -- (d);
    \end{scope}
    \node at ($(d) +(0,-.5)$) {$\vdots$};
    
    \node[align=center] at ($(d)+(0,-2.5cm)$) {any arc\\ orientations};
    
    \foreach \letter in {a,b,c,d} {
      \vertexat{(\letter)}
    }

  \end{tikzpicture}%
   \hspace*{-0cm}and\quad
  \begin{tikzpicture}[baseline={(0,-6)},unav]
    \coordinate (x) at (0,0);
    \coordinate (a) at (-150:3.5cm);
    \coordinate (b) at (-110:3.5cm);
    \coordinate (c) at ( -70:3.5cm);
    \coordinate (d) at ( -30:3.5cm);

    \coordinate (a1) at ($(a) + (0,-3)$);
    \coordinate (b1) at ($(b) + (0,-3)$);
    \coordinate (c1) at ($(c) + (0,-3)$);
    \coordinate (d1) at ($(d) + (0,-3)$);

    \coordinate (a2) at ($(a1) + (0,-3)$);
    \coordinate (b2) at ($(b1) + (0,-3)$);
    \coordinate (d2) at ($(d1) + (0,-3)$);

    \coordinate (a3) at ($(a2) + (0,-3)$);
    \coordinate (b3) at ($(b2) + (0,-3)$);
    \coordinate (d3) at ($(d2) + (0,-3)$);

    \coordinate (b4) at ($(b3) + (0,-3)$);
    \coordinate (d4) at ($(d3) + (0,-3)$);

    \foreach \letter in {a,b,c,d} {
      \draw [lin shorten] (x) -- (\letter);
      \draw [lin shorten]  (\letter) -- (\letter1);
    }

    \foreach \letter in {a,b,d} {
      \draw [lin shorten]  (\letter1) -- (\letter2);
      %\draw [lin shorten]  (\letter2) -- (\letter3);
    }

    \foreach \letter in {d} {
      %\draw [lin shorten]  (\letter3) -- (\letter4);
    }

    \foreach \letter in {a2,b2,c1,d2} {
      \node at ($(\letter) +(0,-.5)$) {$\vdots$};
    }

    \foreach \letter in {x,a,a1,a2,b,b1,b2,c,c1,d,d1,d2} {
      \vertexat{(\letter)}
    }

    \draw [gray,thick,decorate,decoration={brace,amplitude=.3cm}] ($(b2-|d)+(0,-1.1cm)$) -- node [black,below=.3cm] {$\displaystyle \leq \left(\frac{3}{8}+\frac{1}{200}\right)\nn$ branches} ($(b2-|a)+(0,-1.1cm)$);
%    \draw [gray,thick,decorate,decoration={brace,amplitude=.3cm}] ($(d4)+(0,-.8cm)$) -- node[black,below=.3cm,midway] {$\displaystyle\leq \frac{19}{50}n=\frac{3]{8}+\frac{1}{200}$ branches} %
%    ($(d4-|a)+(0,-.8cm)$);
  \end{tikzpicture}%

\end{center}


\bigskip
\newcommand{\yr}[1]{\,\raisebox{1ex}{\fontsize{28}{36}\selectfont\color{dark}\small(#1)}\negthinspace}

{\fontsize{30}{36}\selectfont
  The results above were obtained (in the case of paths)
  by R\'edei\yr{1934}, Thomason\yr{1986}, Havet \& Thomass\'e\yr{2000},
  and (for claws) by Saks \& S\'os\yr{1981} and Lu, Wang \& Wong\yr{1998}.}

\bigskip

\begin{center}
{\itshape\fontsize{40}{40}\selectfont
\emph{An approximate answer and a conjecture}
}
\end{center}

\bigskip

{\color{dark}\scshape Bender \& Wormald (1988)}\newline
\hspace*{1em}$\color{light}\hookrightarrow$ \emph{Most} tournaments contain \emph{most} trees!

\medskip

\makebox{\phantom{\hspace*{1em}$\hookrightarrow$}} \hfill$\mathbb{P}\,(\,$random \textcolor{unav}{tree} in random \textcolor{tournament}{tournament}$\,)\to 1$

\medskip

\hspace*{1em}$\color{light}\hookrightarrow$ \textcolor{light}{Conjecture.} Are most trees unavoidable?

\vfill\columnbreak

\begin{center}
{\itshape\fontsize{40}{40}\selectfont
\emph{Our result}
}
\end{center}

\newcommand{\hook}{\hspace*{1em}$\color{light}\hookrightarrow$ }
\newcommand{\phantomhook}{\makebox{\phantom{\hook}}}

{\color{dark}\scshape Mycroft, N. (2016$^+$)}\newline
\hook Indeed, \emph{most trees are unavoidable!}

\smallskip

\phantomhook \quad$\mathbb{P}\,(\,$random \textcolor{unav}{tree} in \emph{every} \textcolor{tournament}{tournament}$\,)\to 1$ 

\medskip

\hook Obtain sufficient conditions for unavoidability.

\medskip
\hook Available on {\color{dark}\texttt{arxiv.org/abs/1609.03393}}.

\medskip

\begin{center}
{\itshape\fontsize{40}{40}\selectfont
\emph{Ingredients and method}
}
\end{center}

Moon\yr{1968}\,: in a ``typical'' large tree
 few arcs touch each vertex.
\hfill(Maximum degree is sub-logarithmic.)

\bigskip


K\"uhn, Mycroft and Osthus\yr{2011} obtained structural characterization of large tournaments.

\bigskip

Mycroft, N.\yr{2016$^+$}\,

\hook Most trees admit a good ``prunning''.

\hook The prunned tree ``fits'' nicely in any tournament.
(By means of a randomized algorithm, based on
a sharpened version of the result above.)

\hook We may then recover the prunned ends, to obtain a copy of the complete tree!




%green!40!blue


\medskip

\end{multicols}


\begin{tikzpicture}[remember picture, overlay]
  \clip ($(current page.north east) -(.5,2.5)$) -- +(0,1.5cm) arc (60:300:3.5cm) -- cycle;
%  \clip ($(current page.north east) -(2cm,2cm)$) rectangle ($(current page.north west) -(.3cm,0cm)$);
  \node[anchor=north east] (tn) at ($(current page.north east) +(.13,0)$) {\ldots me!};
  
\end{tikzpicture}

%\vspace{.5cm}


\begin{minipage}{.6\textwidth}
  \vfill
  
  \fontsize{16}{18}\selectfont
  Richard Mycroft, \textcolor{dark}{\texttt{r.mycroft@bham.ac.uk}}. School of Mathematics, University of Birmingham, Birmingham, B15 2TT, United Kingdom. Research partially supported by EPSRC grant EP/M011771/1.

  \medskip
  
T\'assio Naia \textcolor{dark}{\texttt{tnaia@member.fsf.org}}. School of Mathematics, University of Birmingham, Birmingham, B15 2TT, United Kingdom. Research supported by CNPq (201114/2004-3). \hfill \textcolor{dark}{\texttt{http://web.mat.bham.ac.uk/T.Naia/}}
\end{minipage}
\begin{minipage}{.38\textwidth}

\begin{center}
%\includegraphics[width=4cm]{pierre}\quad
\raisebox{-1cm}{\includegraphics[width=5cm]{UoB_logo.jpg}}
\quad\includegraphics[width=6cm]{logo-vetorizada_portugues.png}


\end{center}
\end{minipage}

\end{document}
